\documentclass{SeminarV2}

\usepackage[latin1]{inputenc}
\usepackage{amssymb,amsmath,array}
\usepackage{graphicx}
\usepackage{placeins}


%***********************************************************************
% !!!! IMPORTANT NOTICE ON TEXT MARGINS !!!!!
%***********************************************************************
%
% Please avoid using DVI2PDF or PS2PDF converters: some undesired
% shifting/scaling may occur when using these programs
% It is strongly recommended to use the DVIPS converters.
%
% Check that you have set the paper size to A4 (and NOT to letter) in your
% dvi2ps converter, in Adobe Acrobat if you use it, and in any printer driver
% that you could use.  You also have to disable the 'scale to fit paper' option
% of your printer driver.
%
% In any case, please check carefully that the final size of the top and
% bottom margins is 5.2 cm and of the left and right margins is 4.4 cm.
% It is your responsibility to verify this important requirement.  If these margin requirements and not fulfilled at the end of your file generation process, please use the following commands to correct them.  Otherwise, please do not modify these commands.
%
\voffset 0 cm \hoffset 0 cm \addtolength{\textwidth}{0cm}
\addtolength{\textheight}{0cm}\addtolength{\leftmargin}{0cm}

%***********************************************************************
% !!!! USE OF THE SeminarV2 LaTeX STYLE FILE !!!!!
%***********************************************************************
%
% Some commands are inserted in the following .tex example file.  Therefore to
% set up your Seminar submission, please use this file and modify it to insert
% your text, rather than staring from a blank .tex file.  In this way, you will
% have the commands inserted in the right place.

% Edited by Martin Bogdan.

\begin{document}
%style file for Seminar manuscripts
\title{User Manual}

%***********************************************************************
% AUTHORS INFORMATION AREA
%***********************************************************************
\author{Maximilian Joas$^1$
%
% Optional short acknowledgment: remove next line if non-needed
%\thanks{This is an optional funding source acknowledgement.}
%
% DO NOT MODIFY THE FOLLOWING '\vspace' ARGUMENT
\vspace{.3cm}\\
%
% Addresses and institutions (remove "1- " in case of a single institution)
University of Leipzig  - Department of Computer Science \\
Augustusplatz 10, 04109 Leipzig  - Germany\\}

%
% Remove the next three lines in case of a single institution

%***********************************************************************
% END OF AUTHORS INFORMATION AREA
%***********************************************************************

\maketitle

\begin{abstract}


\end{abstract}

\section{multiVitamin}
mulitVitamin is a software package to perform mulitple alignments with graphs.
There are two algorithmns availbale for this task. The Bron Kerbosch \cite{}
and the VF2 algorithmn \cite{}. Details on the algorithmns ca be found in the
theory section.\\
The main function of the package is to allign two or more graphs. This results
in a Newick Tree \cite{} of the best allignment. Sounds exciting, so let's
get started and show how to use this baby.
\section{Installation}

Clone the repo from github with: \\
%git clone https://github.com/MaxJoas/graph_alignment.git
Navigate in the directory where the setup.py file is: \\
%cd multivitamin_project/multivitamin/multivitamin
install with pip3 install -e .
Done now you can use multiVitamin as a command.



\section{Graph Format}

A good first step is to famliarize yourself with the graph format. A graph
file is basically a .txt file, we use the .graph to be more specific.
Every graph consists of nodes and edges. Every node as an id, which is an unique
integer for each node. Optionally nodes can be labelled. The id and label are
seperated by a semicolon. The edges are two connected nodes and represented
by the ids of the two nodes seperated with a semicolon.\\
The file itself is structered as follows: \\
The first line indicates the authors of the graph. The second line shows the number
of node and the third line the number of edges
This is followed by three lines
that indicate whether the nodes / edges are labelled and if the graph is directed.
A blank line indicates the start of the node section, which are represented as
descirbed above. The node section is followed again by a blank line and the edge section.
Let's look at a small example graph for illustration. The graph has 6 nodes
which are all labelled with a 'c' and 6 edged which are not labelled.
Your graphs need to have exactely this format to use this package. expecially
the blank lines are important. For further example graphs go to the root
directory and navigate to the graphs directory. Now that you are familiar with
the graph format, we can take a look how to use the package.
\begin{verbatim}
AUTHOR: Michel K.
#nodes;6
#edges;6
Nodes labelled;True
Edges labelled;False
Directed graph;False

1;c
2;c
3;c
4;c
5;c
6;c

1;2
1;6
2;3
3;4
4;5
5;6
\end{verbatim}

\section{Graph Alignment}









% ****************************************************************************
% BIBLIOGRAPHY AREA
% ****************************************************************************

\begin{footnotesize}
\bibliography{own.bib}
\bibliographystyle{unsrt}
% IF YOU DO NOT USE BIBTEX, USE THE FOLLOWING SAMPLE SCHEME FOR THE REFERENCES
% ----------------------------------------------------------------------------

% ----------------------------------------------------------------------------

% IF YOU USE BIBTEX,
% - DELETE THE TEXT BETWEEN THE TWO ABOVE DASHED LINES
% - UNCOMMENT THE NEXT TWO LINES AND REPLACE 'Name_Of_Your_BibFile'


%\bibliography{Name_Of_Your_BibFile}

\end{footnotesize}

% ****************************************************************************
% END OF BIBLIOGRAPHY AREA
% ****************************************************************************

j\end{document}
